\documentclass[a4paper,12pt]{article}
\usepackage[utf8]{inputenc}
\usepackage[french]{babel}
\usepackage[T1]{fontenc}
\usepackage{amsmath}
\usepackage{amssymb}
\usepackage{color}
\usepackage{amsthm}
\usepackage{listings}
\usepackage{enumitem}
\usepackage{relsize}
\usepackage{dsfont}
\usepackage{graphicx}
\usepackage[margin=2cm]{geometry}
\usepackage{tikz}
\usetikzlibrary{calc}

\newcommand{\K}{\ensuremath\mathbb{K}}
\newcommand{\N}{\ensuremath\mathbb{N}}
\newcommand{\Z}{\ensuremath\mathbb{Z}}
\newcommand{\Q}{\ensureFmath\mathbb{Q}}
\newcommand{\R}{\ensuremath\mathbb{R}}
\newcommand{\U}{\ensuremath\mathbb{U}}
\newcommand{\C}{\ensuremath\mathbb{C}}
\newcommand{\E}{\ensuremath\mathbb{E}}
\newcommand{\V}{\ensuremath\mathbb{V}}
\renewcommand{\P}{\ensuremath\mathbb{P}}

\renewcommand{\(}{\left(}
\renewcommand{\)}{\right)}

\newcommand{\la}{\leftarrow}
\newcommand{\xla}{\xleftarrow}
\newcommand{\ra}{\rightarrow}
\newcommand{\xra}{\xrightarrow}

\renewcommand\labelitemi{---}

%\setlength\parindent{0pt}

\newtheorem*{definition}{Définition}
\newtheorem*{theorem}{Théorème}
\newtheorem*{algo}{Algorithme}
\renewcommand*{\proofname}{Preuve}

\title{Union-Find Blossom algorithm implementation}

\author{Thibaut Pérami, Théophile Wallez}

\begin{document}

\maketitle

\section{Data structure}

Our main structure is an alternating tree which look like this :

\begin{center}
\begin{tikzpicture}
  \tikzstyle{every node}=[draw,shape=circle];
  \node (r) at (0,0) {$r$};
  \node (1) at (2,2) {$1$};
  \node (2) at (2,0) {$2$};
  \node (3) at (2,-2) {$3$};
  \node (1b) at (4, 2) {$1b$};
  \node (2b) at (4, 0) {$2b$};
  \node (3b) at (4,-2) {$3b$};
  \foreach \n in {1,2,3}{
    \draw (r) -- (\n);
  }
  \draw[ultra thick] (1) -- (1b);
  \draw[ultra thick] (2) -- (2b);
  \draw[ultra thick] (3) -- (3b);

  \foreach \n in {1b,2b,3b}{
    \foreach \x in {-0.5,0,0.5}{
      \node (tmp) at ([shift=({2,\x})]\n) {$$};
    \draw (\n) -- (tmp);
    }
  }
\end{tikzpicture}
\end{center}

The tree is included in the graph. The bold edge are the the edge in the
matching. The nodes that are at an odd distance from the root are called
\emph{before} nodes and those at even distance are called \emph{after} nodes

We also have an union-find structure to represent nodes in the contracted graph: for a
given node in the original graph, its class in the union-find is the
corresponding node in the contracted graph. In practice there is no rank
constraint optimization in the union-find because, we need the force which is
the representant of each class (see section 3)

Thus we maintain at the same time two trees, the tree in the contracted graph and the
tree in the original graph. Only the tree in the contracted graph is an
alternating tree (before node are contracted in after node and thus the will
have edge not in matching going out of them).
In practice, only the tree in original graph is
stored and the tree in contracted graph is deduced with the union find. The tree
is simply stored by remembering for each node which is its parent node. It is
thus impossible to move down in the tree. If you want to go up in the parent
tree you just need to take the representant of the parent node.

\section{Main algorithm}

While exploring we maintain a list of after node to be explored. The main
algorithm consist of taking a node in this list and exploring it until either an
augmenting path is found or the list is empty ans thus there exists no
augmenting path starting from $r$.


During the exploration of a after node we look at all its neighbors, there are
some possibilities:
\begin{itemize}
\item Both nodes are in the same contracted node : ignore
\item The neighbor has already been visited as a before node : we do nothing
\item The neighbor has already been visited as an after node : we have found a
  cycle, see section 3.
\item The neighbor has never been visited and is matched : we add this new node
  and its match to the tree and we add the matching node to the todo list.
\item The neighbor has never been visited and is not matched : we have found an
  augmenting path, we can trivially have an augmenting path in the contracted
  graph by going up in the tree but we want one in the original graph. (see in
  section 4 to see how we do it)
\end{itemize}



\section{Contracting}

An odd cycle in the alternating tree (in the contracted graph) is defined as a
common ancestor (here $r$) and a base edge (here $1b - 2b$). The base edge is
not part of the alternating tree, but all the other edges of cycle are.
When contracting this cycle, we defined the representant of the cycle to be
exactly $r$ :
\begin{center}
\begin{tikzpicture}
  \tikzstyle{every node}=[draw,shape=circle];
  \node (r) at (0,0) {$r$};
  \draw[ultra thick] (-2,0) -- (r);
  \node (1) at (2,1) {$1$};
  \node (2) at (2,-1) {$2$};
  \node (1b) at (4, 1) {$1b$};
  \node (2b) at (4, -1) {$2b$};
  \foreach \n in {1,2}{
    \draw (r) -- (\n);
  }
  \draw[ultra thick] (1) -- (1b);
  \draw[ultra thick] (2) -- (2b);
  \draw[dashed] (1b) -- (2b);
\end{tikzpicture}
\end{center}

In practice, a contracted node in the contracted alternating tree is always in a
position of after-node. Before-nodes are always original nodes even in the
contracted tree. Thus when going up in the contracted tree from an after-node,
we do not need to explicitly take the representant of the parent, just the
parent is sufficient.


Concretely, during contraction we firstly need to find the common ancestor and
then:
\begin{itemize}
\item We need to tell all the vertex of the cycle that they are represented by
  $r$.
\item We need to add any before node that are members of the cycle
to the todo list (here $1$ and $2$) because there non matched edge were not
explored until the contraction but are are now outgoing edge of the contracted
node represented by $r$ which is an after-node.
\item We need to store for each before-node of the cycle which is the base edge
  of cycle for decontraction (cf section 4).
\end{itemize}

\section{Augmenting}



Cas de profondeur 2 et genéralisation + dessin.

\section{Initialization}

Théophile






\end{document}
