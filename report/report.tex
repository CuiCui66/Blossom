\documentclass[a4paper,12pt]{article}
\usepackage[utf8]{inputenc}
\usepackage[french]{babel}
\usepackage[T1]{fontenc}
\usepackage{amsmath}
\usepackage{amssymb}
\usepackage{amsthm}
\usepackage{listings}
\usepackage{enumitem}
\usepackage{relsize}
\usepackage{dsfont}
\usepackage{graphicx}
\usepackage[margin=2cm]{geometry}
\usepackage{tikz}
\usetikzlibrary{calc}

\newcommand{\K}{\ensuremath\mathbb{K}}
\newcommand{\N}{\ensuremath\mathbb{N}}
\newcommand{\Z}{\ensuremath\mathbb{Z}}
\newcommand{\Q}{\ensureFmath\mathbb{Q}}
\newcommand{\R}{\ensuremath\mathbb{R}}
\newcommand{\U}{\ensuremath\mathbb{U}}
\newcommand{\C}{\ensuremath\mathbb{C}}
\newcommand{\E}{\ensuremath\mathbb{E}}
\newcommand{\V}{\ensuremath\mathbb{V}}
\renewcommand{\P}{\ensuremath\mathbb{P}}

\renewcommand{\(}{\left(}
\renewcommand{\)}{\right)}

\newcommand{\la}{\leftarrow}
\newcommand{\xla}{\xleftarrow}
\newcommand{\ra}{\rightarrow}
\newcommand{\xra}{\xrightarrow}

\renewcommand\labelitemi{---}

%\setlength\parindent{0pt}

\newtheorem*{definition}{Définition}
\newtheorem*{theorem}{Théorème}
\newtheorem*{algo}{Algorithme}
\renewcommand*{\proofname}{Preuve}

\title{Union-Find Blossom algorithm implementation}

\author{Thibaut Pérami, Théophile Wallez}

\begin{document}

\maketitle

\section{Data structure}

Arbre alternant + dessin

\begin{tikzpicture}
  \tikzstyle{every node}=[draw,shape=circle];
  \node (r) at (0,0) {$r$};
  \node (1) at (2,2) {$1$};
  \node (2) at (2,0) {$2$};
  \node (3) at (2,-2) {$3$};
  \node (1b) at (4, 2) {$1b$};
  \node (2b) at (4, 0) {$2b$};
  \node (3b) at (4,-2) {$3b$};
  \foreach \n in {1,2,3}{
    \draw (r) -- (\n);
  }
  \draw[ultra thick] (1) -- (1b);
  \draw[ultra thick] (2) -- (2b);
  \draw[ultra thick] (3) -- (3b);

  \foreach \n in {1b,2b,3b}{
    \foreach \x in {-0.5,0,0.5}{
      \node (tmp) at ([shift=({2,\x})]\n) {$$};
    \draw (\n) -- (tmp);
    }
  }
\end{tikzpicture}

Notion de sommet before et after

Notion d'arête de base + dessin.

Ordre d'exploration


\section{Contracting}

Recherche ancêtre commun

union-find

reexploration des sommet before.

\section{Augmenting}

Exemple sur cycle simple + dessin

Cas de profondeur 2 et genéralisation + dessin.

\section{Initialization}

Théophile






\end{document}
